\section{Sinus}

\subsection{main.c}
\begin{lstlisting}[caption={Activate necessary timers},label={lst:Registers}]
  	HAL_TIM_OC_Start(&htim9, TIM_CHANNEL_1);
  	HAL_TIM_OC_Start(&htim10, TIM_CHANNEL_1);
  	HAL_TIM_PWM_Start(&htim3, TIM_CHANNEL_1);
  	HAL_TIM_PWM_Start(&htim3, TIM_CHANNEL_2);
  	HAL_TIM_PWM_Start(&htim3, TIM_CHANNEL_3);
\end{lstlisting}

\begin{lstlisting}[caption={Start Interrupts},label={lst:Registers}]
  	HAL_TIM_Base_Start_IT(&htim9);
  	HAL_TIM_Base_Start_IT(&htim10);
\end{lstlisting}

\begin{lstlisting}[caption={Fill sintab array},label={lst:Registers}]
	for (int i = 0; i < AANTAL_TIJDSTAPPEN; i++) {
		sintab[i] = sin(i*2*M_PI/AANTAL_TIJDSTAPPEN) * 0.5 + 0.5;
	}
\end{lstlisting}

\begin{lstlisting}[caption={Enable Gate Drivers},label={lst:Registers}]
	GPIOC->ODR = 0xE000;
\end{lstlisting}

\begin{lstlisting}[caption={Read Potentiometer data from ADC for RPM control},label={lst:Registers}]
	HAL_ADC_Start(&hadc1);
	HAL_ADC_PollForConversion(&hadc1, HAL_MAX_DELAY);
	WantedRPM = HAL_ADC_GetValue(&hadc1);
\end{lstlisting}

\begin{lstlisting}[caption={Keep RPM to the minimum set in main.h},label={lst:Registers}]
	if (WantedRPM < MinimumRPM) {
		WantedRPM = MinimumRPM;
	}
\end{lstlisting}

\begin{lstlisting}[caption={Set PSC to appropriate value for RPM},label={lst:Registers}]
	TIM10->PSC = (15 * Fapb2clk) / (256 * WantedRPM) - 1;
\end{lstlisting}

\begin{lstlisting}[caption={Send RPM data to PC},label={lst:Registers}]
	len = snprintf(buf, sizeof(buf), "\n\rWanted RPM / Current RPM: %lu / %lu", WantedRPM, CurrentRPM);
	CDC_Transmit_FS((uint8_t *) buf, len);
	HAL_Delay(10);
\end{lstlisting}

\begin{lstlisting}[caption={i2c instance},label={lst:Registers}]
 hi2c2.Instance = I2C2;
  hi2c2.Init.ClockSpeed = 100000;
  hi2c2.Init.DutyCycle = I2C_DUTYCYCLE_2;
  hi2c2.Init.OwnAddress1 = 0;
  hi2c2.Init.AddressingMode = I2C_ADDRESSINGMODE_7BIT;
  hi2c2.Init.DualAddressMode = I2C_DUALADDRESS_DISABLE;
  hi2c2.Init.OwnAddress2 = 0;
  hi2c2.Init.GeneralCallMode = I2C_GENERALCALL_DISABLE;
  hi2c2.Init.NoStretchMode = I2C_NOSTRETCH_DISABLE;
  if (HAL_I2C_Init(&hi2c2) != HAL_OK)
  {
    Error_Handler();
  }
\end{lstlisting}

\begin{lstlisting}[caption={SPI instance},label={lst:Registers}]
  hspi1.Instance = SPI1;
  hspi1.Init.Mode = SPI_MODE_MASTER;
  hspi1.Init.Direction = SPI_DIRECTION_2LINES;
  hspi1.Init.DataSize = SPI_DATASIZE_8BIT;
  hspi1.Init.CLKPolarity = SPI_POLARITY_LOW;
  hspi1.Init.CLKPhase = SPI_PHASE_1EDGE;
  hspi1.Init.NSS = SPI_NSS_SOFT;
  hspi1.Init.BaudRatePrescaler = SPI_BAUDRATEPRESCALER_2;
  hspi1.Init.FirstBit = SPI_FIRSTBIT_MSB;
  hspi1.Init.TIMode = SPI_TIMODE_DISABLE;
  hspi1.Init.CRCCalculation = SPI_CRCCALCULATION_DISABLE;
  hspi1.Init.CRCPolynomial = 10;
  if (HAL_SPI_Init(&hspi1) != HAL_OK)
  {
    Error_Handler();
  }
\end{lstlisting}

\begin{lstlisting}[caption={TIM3 setup},label={lst:Registers}]
 /* USER CODE END TIM3_Init 1 */
  htim3.Instance = TIM3;
  htim3.Init.Prescaler = 0;
  htim3.Init.CounterMode = TIM_COUNTERMODE_UP;
  htim3.Init.Period = 1919;
  htim3.Init.ClockDivision = TIM_CLOCKDIVISION_DIV1;
  htim3.Init.AutoReloadPreload = TIM_AUTORELOAD_PRELOAD_DISABLE;
  if (HAL_TIM_PWM_Init(&htim3) != HAL_OK)
  {
    Error_Handler();
  }
  sMasterConfig.MasterOutputTrigger = TIM_TRGO_RESET;
  sMasterConfig.MasterSlaveMode = TIM_MASTERSLAVEMODE_DISABLE;
  if (HAL_TIMEx_MasterConfigSynchronization(&htim3, &sMasterConfig) != HAL_OK)
  {
    Error_Handler();
  }
  sConfigOC.OCMode = TIM_OCMODE_PWM1;
  sConfigOC.Pulse = 0;
  sConfigOC.OCPolarity = TIM_OCPOLARITY_HIGH;
  sConfigOC.OCFastMode = TIM_OCFAST_DISABLE;
  if (HAL_TIM_PWM_ConfigChannel(&htim3, &sConfigOC, TIM_CHANNEL_1) != HAL_OK)
  {
    Error_Handler();
  }
  if (HAL_TIM_PWM_ConfigChannel(&htim3, &sConfigOC, TIM_CHANNEL_2) != HAL_OK)
  {
    Error_Handler();
  }
  if (HAL_TIM_PWM_ConfigChannel(&htim3, &sConfigOC, TIM_CHANNEL_3) != HAL_OK)
  {
    Error_Handler();
  }
  /* USER CODE BEGIN TIM3_Init 2 */

  /* USER CODE END TIM3_Init 2 */
  HAL_TIM_MspPostInit(&htim3);

}
\end{lstlisting}

\begin{lstlisting}[caption={Calculate RPM and reset counter},label={lst:Registers}]
	CurrentRPM = 600 * (RPM / 6.0f);
	RPM = 0;
\end{lstlisting}

\begin{lstlisting}[caption={Disable Gate Drivers if RPM is too high},label={lst:Registers}]
	if (CurrentRPM > MaximumRPM) {
		GPIOC->ODR &= 0x1FFF;
	}
\end{lstlisting}

\begin{lstlisting}[caption={Write some registers},label={lst:Registers}]
// Set PWM timers to next step in sinusoid generation
	TIM3->CCR1 = TIM3ARR * sintab[ (j + OffsetU) % AANTAL_TIJDSTAPPEN];
	TIM3->CCR2 = TIM3ARR * sintab[ (j + OffsetV) % AANTAL_TIJDSTAPPEN];
	TIM3->CCR3 = TIM3ARR * sintab[ (j + OffsetW) % AANTAL_TIJDSTAPPEN];

	j++;
\end{lstlisting}

\begin{lstlisting}[caption={Write some registers},label={lst:Registers}]
	if( j > AANTAL_TIJDSTAPPEN) {
		j = 0; // Reset j when full sinusoid has been made.
	}
\end{lstlisting}

\section{Commutation}
\subsection{main.c}
\begin{lstlisting}[caption={Write some registers},label={lst:Registers}]
uint8_t HallSensor = 0;
uint16_t Commutation[6][3] = {
		{0x0, 0xF, 0xF},
		{},
		{},
		{},
		{},
		{}
};
\end{lstlisting}

\begin{lstlisting}[caption={Write some registers},label={lst:Registers}]
HAL_StatusTypeDef StartupSequence(char Direction);
HAL_StatusTypeDef PrepareCommutation(void);
\end{lstlisting}

\begin{lstlisting}[caption={Write some registers},label={lst:Registers}]
  TIM1->CR2 |= 0x0005; 			// Set CCPC 1 and CCUS 1 in CR2
  TIM1->SR &= ~TIM_SR_COMIF; 	// Set COMIF low in SR to not trigger commutation event
  TIM1->DIER |= TIM_DIER_COMIE; // Enable Commutation events in DIER register

  StartupSequence('F');
\end{lstlisting}

\begin{lstlisting}[caption={Set first commutation state according to Hall sensors},label={lst:Registers}]
HAL_StatusTypeDef StartupSequence(char Direction) {

  if (PrepareCommutation() == HAL_ERROR) {
	  return HAL_ERROR;
  }

  // Start HallSensor timer
  HAL_TIMEx_HallSensor_Start(&htim2);

  // Start all PWM signals on TIM1
  HAL_TIM_PWM_Start(&htim1, TIM_CHANNEL_1);
  HAL_TIM_PWM_Start(&htim1, TIM_CHANNEL_2);
  HAL_TIM_PWM_Start(&htim1, TIM_CHANNEL_3);

  // Start Interrupts
  HAL_TIM_Base_Start_IT(&htim2);
  HAL_TIM_Base_Start_IT(&htim1);

  // Set COMG bit in EGR for first commutation
  TIM1->EGR |= TIM_EGR_COMG;

  return HAL_OK;

}

HAL_StatusTypeDef PrepareCommutation() {

  // Read IDR for Hall Sensor status
  uint8_t Hall = GPOIA->IDR & 0x0007;

  switch (Hall) {
  case 0b001:
	  HallSensor = 0;
  break;
  }

  return HAL_OK;

}
\end{lstlisting}



\subsection{stm32f4xx\textunderscore it.c}
\begin{lstlisting}[caption={Set new Commutation states CCxE, CCxNE, OCxM in CCER and CCMR1 and CCMR2 according to array},label={lst:Registers}]
void TIM1_TRG_COM_TIM11_IRQHandler(void)
{
  /* USER CODE BEGIN TIM1_TRG_COM_TIM11_IRQn 0 */

	// Set new Commutation states CCxE, CCxNE, OCxM in CCER and CCMR1 and CCMR2 according to array
	// Reset COMIF in SR register


  /* USER CODE END TIM1_TRG_COM_TIM11_IRQn 0 */
  HAL_TIM_IRQHandler(&htim1);
  /* USER CODE BEGIN TIM1_TRG_COM_TIM11_IRQn 1 */

  /* USER CODE END TIM1_TRG_COM_TIM11_IRQn 1 */
}

/**
  * @brief This function handles TIM2 global interrupt.
  */
void TIM2_IRQHandler(void)
{
  /* USER CODE BEGIN TIM2_IRQn 0 */

  // Set COMG bit in EGR
  TIM1->EGR |= TIM_EGR_COMG;

  /* USER CODE END TIM2_IRQn 0 */
  HAL_TIM_IRQHandler(&htim2);
  /* USER CODE BEGIN TIM2_IRQn 1 */

  /* USER CODE END TIM2_IRQn 1 */
}
\end{lstlisting}
